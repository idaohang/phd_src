В первой главе рассматривается сигнал с расширенным спектром полученный методом "прямой последовательности".
Данный метод заключается в том, что гармоническая несущая сигнала модулируется высокоскоростным (широкополосным)
расширяющим сигналом \cite{sklyar}. Методы генерации таких последовательностей рассмотрены, например, в \cite{gantmaher-book, varakin-book, pestryakov-book}.
В данной работе в качестве ПСП используется код Голда. Свойства данного семейства ПСП подробно рассмотрены в \cite{gold-ieee},
а так же краткое описание свойств без доказательства приведены в \cite{tsui, akos-book}. Метод генерирования ПСП подробно рассмотрен
во многих источниках \cite{tsui, akos-book, kaplan} и в данной работе рассматриваться не будет.

В данном типе модуляции информационные сообщения кодируются изменением фазы несущей сигнала.
Математическую модель сигнала спутниковой системы навигации (СНС) Navstar GPS можно представить формулой \cite{tsui, akos-book, kaplan}:

\begin{equation}
	\label{eq:cdma_eq}
	s_k(t)=D_k(t)C_k(t)\cos{(\omega_{k}t + \phi_k(t))} + n_k(t)
\end{equation}
где ${D_k}$- информационный бит, ${C_k}$ - расширяющий код, ${\phi_k(t)}$ - фаза обусловленная допплеровским смещением частоты, а так же шумом осциллятора
и ${n_k(t)}$ - шумовая компонента. Формула  \ref{eq:cdma_eq} представляет математическую модель сигнала для спутника номер ${k}$.
Всего в группировке спутников Navstar GPS доступно 32 спутника. После повторного модулирования сигнала, указанного в выражении \ref{eq:cdma_eq},
получается:

\begin{equation}
	\label{eq:cdma_strip_eq}
	x_k(m)=\cos{(\omega_{k}m + \phi_k(m))} + n_k(m)
\end{equation}
Информационный бит принят за константу, так как в случае смены бита гармонический характер сигнала нарушается и детектирование становится невозможным.
