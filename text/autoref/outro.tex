
{\bf{В заключении}}
отражены результаты работы и обозначены направления дальнейшего исследования

\noindent\centerline{\bf{Основные результаты и выводы}}

В итоге проведенных в диссертационной работе теоритических и экспериментальных исследований получены следующие основные результаты:

\begin{enumerate}
	\item Разработан алгоритм на основе параметрического метода оценки информационных для одного источника с CDMA-сигналом на фоне аддитивного белого гауссового шума при отстутствии МКИ.
		Получены значение вычислительной сложности и точности оценки частоты. Алгоритм на основе праметрического метода оценки частоты позволяет снизить вычислительные
		затраты в 1.5 раза в сравнении с типовым подходом при этом точность оценки, при отсутствии МКИ, не привосходит нескольких десятков герц для сигналов
		с ОСШ более 25 дБ.
	\item Усовершенствован алгоритм итеративного вычисления автокорреляционной функции. Получены значение вычислительной сложности и оценка
		прироста ОСШ в зависимости от количества итераций. Алгоритм позволяет увеличить ОСШ на 20 дБ при вычислении трех итераций
		пересчета автокорреляционной функции, в то же время вычислительная сложность, сниженная с квадратичной до логарифмической,
		позволяет использовать его в приемниках реального времени.
	\item Разработан комплексированный алгоритм оценки информационных параметров CDMA-сигнала на основе алгоритма Delay and Multiply Approach с использованием
		предложенного усовершенствованного итеративного алгоритма вычисления автокорреляционной функции и параметрического
		метода оценки частоты на фоне аддитивного белого гауссового шума при наличии МКИ. Алгоритм позволяет получить оценку частоты, удовлетворяющую
		допустимой входной расстройке ФАПЧ, для значений ОСШ сигнала порядка -20 дБ без накопления, при вычислительной сложности в 3 раза меньшей в сравнении с традиционным подходом.
	\item Результаты исследований с использованием имитационного моделирования в математическом пакете MATLAB, а также полунатурного моделирования,
		на разработанной программно-аппаратной платформе и сигнале, полученном из внешних источников, согласуются и подтверждают возможность
		использования разработанного комплексированного алгоритма оценки информационных параметров CDMA-сигнала в системе Navstar GPS у уровнем ОСШ порядка -27 дБ.
	\item Предложенный комплексированный алгоритм оценки информационных параметров CDMA-сигнала позволяет добиться наилучших результатов при
		типовых значениях допплеровского смещения и типовых сценариях распространения сигнала в наземных пользовательских приемниках.
\end{enumerate}

\paragraph{Основные результаты диссертации отражены в работах:}
\begin{enumerate}
	{\bf{
	\item Сидоркина Ю.А., Никифоров А.А. Алгоритм оценки параметров широкополосного сигнала на ограниченном интервале наблюдения //
		Научный вестник МГТУ ГА. 2014. №209 (в публикации)
	\item Никифоров А.А. Оптимизация алгоритма последовательного вычисления автокорреляционной функции //
		Образование. Наука. Научные кадры. 2013 №5. С. 204-207.
	\item Никифоров А.А. Алгоритм итеративного вычисления автокорреляционной функции в задаче оценки частоты широкополосного сигнала //
		Механизация строительства. 2013. № 11. С. 53-55.
	\item Никифоров А.А. Мельников А.О. Токарев С.В. "Детектирование сигналов с расширенным спектром на основе АР модели,
		Промышленные АСУ и контроллеры", 2013. №5 2013, С. 51-54. (п.л. 0.1875 / п.л. 0.1)
	}}

	\item Никифоров А.А., Применение алгоритма Delay and Multiply Approach и АР модели для обнаружения и оценки параметров ШПС //
		доклады 7-ой Всероссийской конференции «Радиолокация и радиосвязь» в рамках Московской Микроволновой Недели. 2013. С. 223-227 
	\item Никифоров А.А. Детектирование сигналов с расширенным спектром на основе АР модели с учетом мощности шума // Доклады международной конференции
		«Радиоэлектронные устройства и системы для инфокоммуникационных технологий - РЕС-2013», М. 2013. Выпуск LXVIII. С. 139-143
	\item Никифоров А.А., Мельников А.О. Методы детектирования систем спутниковой навигации // Интеллектуальный потенциал XXI века:
		ступени познания: Сборник мат-в V Международной студенческой научно-практической конференции. 2011 Часть 2. С. 67 - 70. (02 п.л.)

	\item Никифоров А.А. Создание лабораторного стенда для приема сигналов спутниковых систем навигации // ВЕСТНИК Молодых ученых Московского
		государственного университета приборостроения и информатики. М. 2011. №9. C. 55-66
\end{enumerate}
