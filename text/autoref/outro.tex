
{\bf{В заключении}}
отражены результаты работы и обозначены направления дальнейшего исследования

\noindent\centerline{\bf{Основные результаты и выводы}}
В ходе диссертационного исследования получены следующие результаты:
\begin{enumerate}
\item Разработан алгоритм на основе параметрического метода оценки частоты для одного источника с широкополосным сигналом.
\item Усовершенствован алгоритм итеративного вычисления автокорреляционной функции, что позволяет использовать его в приемниках
	реального времени.
\item Разработан алгоритм оценки параметров широкополосного сигнала на основе алгоритма Delay and Multiply Approach с использованием
	предложенного усовершенствованного итеративного алгоритма вычисления автокорреляционной функции и параметрического
	метода оценки частоты. Данное решение имеет более высокую точность оценки в сравнении с традиционным
	параллельным коррелятором, в то же время оценка параметра может быть получена за меньшее количество итераций.
\item Произведено имитационное моделирование предложенного алгоритма для проверки положений, выносимых на защиту.
\item Произведено обоснование актуальности и возможности применения параметрического метода оценки частоты для сигналов
	с расширенным спектром.
\item Отражены возможные направления дальнейших исследований в области применения параметрического анализа в системах
	с расширенным спектром.
\end{enumerate}

\paragraph{Публикации}
\begin{enumerate}
	\item {\underline{Статьи в журналах, рекомендованных ВАК}}
	\begin{enumerate}
		\setcounter{enumi}{1}
		\item Никифоров А.А., "Оптимизация алгоритма последовательного вычисления автокорреляционной функции".
			Образование. Наука. Научные кадры., №5, 2013 сс 204-207.
		\item Никифоров А.А.,"Алгоритм итеративного вычисления автокорреляционной функции в задаче оценки частоты широкополосного сигнала".
			Механизация строительства, Ноябрь, 2013 (в публикации)
		\item Никифоров А.А. Мельников А.О. Токарев С.В. "Детектирование сигналов с расширенным спектром на основе АР модели,
			Промышленные АСУ и контроллеры", Май 2013, стр. 51-54.
	\end{enumerate}

	\item {\underline{Материалы докладов на конференциях}}
	\begin{enumerate}
		\item Никифоров А.А., Применение алгоритма Delay and Multiply Approach и АР модели для обнаружения и оценки параметров ШПС. 
			доклады 7-ой Всероссийской конференции «Радиолокация и радиосвязь» в рамках Московской Микроволновой Недели, стр 223 
		\item Никифоров А.А. "Детектирование сигналов с расширенным спектром на основе АР модели с учетом мощности шума", Международная конференция
			"Радиоэлектронные устройства и системы для инфокоммуникационных технологий - РЕС-2013", Москва, Россия, доклады. Выпуск LXVIII. стр. 139.
		\item Никифоров А.А., Мельников А.О. "Методы детектирования систем спутниковой навигации". Интеллектуальный потенциал XXI века:
			ступени познания: Сборник мат-в V Международной студенческой научно-практической конференции: в 2-х частях. Часть 2 / Под общ. Ред.
			С.С.Чернова. – Новосибирск: Издательство НГТУ, 2011. – СС. 66 - 70. – 0, 2 п.л.
	\end{enumerate}

	\item {\underline{Статьи в сборниках статей}}
	\begin{enumerate}
		\item Никифоров А.А.Создание лабораторного стенда для приема сигналов спутниковых систем навигации. ВЕСТНИК Молодых ученых Московского
			государственного университета приборостроения и информатики, Выпуск №9. Москва, 2011. – cc 55-66. ISBN 978-5-8068-0484-7
	\end{enumerate}
\end{enumerate}
