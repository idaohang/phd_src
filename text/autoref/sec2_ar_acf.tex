\paragraph{Вторая глава} содержит разделы о применении методов параметрического спектрального оценивания и автокоррелляционном анализе в приложении к оценке
параметров широкополосного сигнала. В данной главе приводится обоснование эффективности применения АР модели для оценки параметров широкополосного сигнала,
а так же рассматриваются особенности работы с широкополосным сигналом. При оценке параметров широкополосного сигнала стоит учитывать,
что поиск производится в двумерной области неопределенности: неопределенность по фазе ПСП и неопределенность по частоте.
АР модель может быть применена только для гармонического сигнала. Таким образом для применения АР модели сигнал необходимо повторно модулировать ПСП
с заранее известной фазой для восстановления гармонической компоненты.

Особенностью широкополосного сигнала является наличие ярко выраженного одиночного пика в спектральной области сигнала после повторной модуляции ПСП. 
Этот факт дает основания полагать, что такие сигналы могут быть достаточно точно представлены с помощью АР модели второго порядка.
Использование процедуры оценки параметров АР модели вместо перебора по заранее заданным значениям частоты позволяет вести поиск сигнала
в широком диапазоне частот и определять сдвиг с более высокой точностью, устраняя необходимость дополнительного уточнения
доплеровского сдвига перед запуском процедуры сопровождения сигнала.

Точность оценок, полученных на основе АР модели быстро снижается при наличии сильного или окрашенного шума. Для преодоления указанных
трудностей в данной работе предлагается использовать процедуру многократной переоценки АКФ. Процедура переоценки АКФ позволяет
повысить ОСШ и привести спектр сигнала с симметричному виду. Симметричный спектр гармонического сигнала позволяет произвести
достаточно точную оценку промежуточной частоты входного гармонического сигнала.

Так же во втором разделе рассмотрено неравенство Крамера-Рао. Данное неравенство является очень важным при оценке параметров любым
оценщиком, так как дает возможность вычислить базу оценки эффективности разрабатываемого алгоритма оценки. В данной главе
приводится неравенство Крамера-Рао для оценки частоты гармонического сигнала, что позволяет определить эффективность
предлагаемых решений.
