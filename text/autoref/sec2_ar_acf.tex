\paragraph{Вторая глава} содержит разделы о параметрическом спектральном оценивании и автокоррелляционном анализа в приложении к детектированию и оценке
частоты сигнала в широкополосных системах. В данной главе приводится обоснование эффективности применения АР-модели для оценки параметра ШПС.

Особенностью ШПС является наличие ярко выраженного одиночного пика в спектральной области сигнала после снятия расширяющего кода. 
Этот факт дает основания полагать, что такие сигналы могут быть достаточно точно представлены с помощью АР модели второго порядка.
Использование процедуры оценки параметров АР модели вместо перебора по заранее заданным значениям позволяет вести поиск сигнала
в широком диапазоне частот и определять сдвиг с более высокой точностью, устраняя необходимость дополнительного уточнения
доплеровского сдвига перед запуском процедуры сопровождения сигнала.

Точность оценок, полученных на основе АР модели быстро снижается при наличии сильного или окрашенного шума. Для преодоления указанных
трудностей в данной работе предлагается использовать процедуру многократной переоценки АКФ. 
