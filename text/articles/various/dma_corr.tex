%\documentclass[a4paper,12pt]{eskdtext} %размер бумаги устанавливаем А4, шрифт 12пунктов
\documentclass[a4paper,12pt]{scrartcl} %размер бумаги устанавливаем А4, шрифт 12пунктов
%\documentclass[14pt,a4paper,twoside]{report}
\usepackage[T2A]{fontenc}
\usepackage[utf8]{inputenc} %включаем свою кодировку: koi8-r или utf8 в UNIX, cp1251 в Windows
\usepackage[english,russian]{babel} %используем русский и английский языки с переносами
\usepackage{amssymb,amsfonts,amsmath,mathtext,cite,enumerate,float} %подключаем нужные пакеты расширений
\usepackage[dvips]{graphicx} %хотим вставлять в диплом рисунки?
%\usepackage{soul}
\usepackage[12pt]{extsizes}
\usepackage{longtable}
\usepackage{graphicx}
\usepackage{indentfirst}
\usepackage{enumerate}

\usepackage[format=plain,labelformat=simple,labelsep=endash,figurename=\CYRR\cyri\cyrs\cyru\cyrn\cyro\cyrk]{caption}
%%%%%%%%%%%%%%%%%%%%%%%%%%%%%%

\begin{document}

\paragraph{Интерференция в алгоритме Delay and Multiply Approach.}

Обозначим сигнал как ${s(t)=A C(t) \exp^{j \omega t}}$, тогда смесь двух сигналов и АБГШ может быть записана как (фаза для простоты не рассматривается):
\begin{equation}
	x(t) = A_1 C_1(t) \exp^{j \omega_1 t} + A_2 C_2(t) \exp^{j \omega_2 t} + n(t)
\end{equation}

Алгоритм Delay and Multiply Approach (DMA) представляет собой произведение смеси и задерженной на ${\tau}$ копии смеси. Обозначим за ${C(t)}$ - псевдо-случайную
последовательность (ПСП). Примем во внимание свойство ПСП, что произведение двух ПСП при смещении кратном длинне чипа (${\tau = kl}$), где ${k}$ - целое число,
а ${l}$ - длина чипа (при учете апсемплинга она больше 1) представляет собой новую ПСП, с такими же корреляционными свойствами.

Учитывая вышесказанное, можно записать:
\begin{eqnarray}
	x(t)x(t-\tau)^* & = & 	\left( A_1 C_1(t) \exp^{j \omega_1 t} + A_2 C_2(t) \exp^{j \omega_2 t} + n(t) \right) \\
		        &   &	\left( A_1 C_1(t) \exp^{-j \omega_1 (t-\tau)} + A_2 C_2(t) \exp^{-j \omega_2 (t-\tau)} + n(t-\tau) \right)  =  \nonumber  \\
	 A_1^2 C_{11}(t)\exp^{j \omega_1 \tau} & + & A_1 A_2 C_1(t)C_2(t-\tau) \exp^{j(\omega_1 t - \omega_2 t + \omega_2 \tau)} + A_1 C_1(t)\exp^{j \omega_1 t} n(t-\tau) + \nonumber \\
	 A_2^2 C_{22}(t)\exp^{j \omega_2 \tau} & + & A_1 A_2 C_1(t-\tau)C_2(t) \exp^{j(\omega_2 t - \omega_1 t + \omega_1 \tau)} + A_2 C_2(t) \exp^{j \omega_2 t}n(t-\tau) + \nonumber \\
			& + & n(t) A_1 C_1(t-\tau)\exp^{-j \omega_1 (t-\tau)} + n(t) A_2 C_2(t-\tau)\exp^{-j \omega_2 (t-\tau)} + n(t)n(t-\tau) \nonumber
\end{eqnarray}

Термы ${A_1^2 C_{11}(t)\exp^{j \omega_1 \tau}}$ и ${A_2^2 C_{22}(t)\exp^{j \omega_2 \tau}}$ являются константам, фактически состоящими из ПСП, умноженного на константу.

Термы ${A_1 A_2 C_1(t)C_2(t) \exp^{j(\omega_2 t - \omega_1 t + \omega_1 \tau)}}$ и ${A_1 A_2 C_1(t)C_2(t) \exp^{j(\omega_1 t - \omega_2 t + \omega_2 \tau)}}$ являются
колебаниями, но их частота существенно ниже частот в области неопределенности, а значит они могут быть отфильтрованы. Следует отметить, что начальная фаза ПСП неизвестна
и свойства полученной ПСП ${C_1(t)C_2(t-\tau)}$ сильно зависят от ${\tau}$, если ${\tau}$ сдвинута на полчипа, новый код перестает быть последовательностью максимальной длинны,
а значит обладает плохими корреляционными свойствами относительно других ПСП, по которым может производиться перебор.

Там хоть и низкая частота, но есть ПСП. В целом думаю можно отсечь. Нужно помоделировать. В плане базового вопроса - высокие значения кросс-корреляции пока не просматривается.

\end{document}
