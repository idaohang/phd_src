\addcontentsline{toc}{section}{ВЫВОДЫ и ЗАКЛЮЧЕНИЕ}
\section*{ВЫВОДЫ и ЗАКЛЮЧЕНИЕ}

В данной работе предложено использоваться параметрический метод оценки частоты в задаче оценки и детектирования широкополосного сигнала. Разработано 2 алгоритма
оценки параметров широкополосного сигнала, проанализировано качество оценки частоты в зависимости от ОСШ. Разработан эффективный алгоритм
увеличения ОСШ в оценке АКФ, позволяющий применять его для обработки сигналов в режиме реального времени.
Данный алгоритм позволяет использовать АР модель для достаточно точной оценки частоты сигнала при низких значениях ОСШ. Без применения данного алгоритма
оценка, получаемая с использованием АР модели получается сильно смещенная.

Применение комбинации алгоритмов Delay and Multiply Approach, усовершенствованного
алгоритма оценки АКФ и алгоритма оценки частоты с использованием АР модели второго порядка дает качественный синергетический эффект. Что позволяет
расширить границы применения каждого из алгоритмов и получить эффективный, с точки зрения вычислительных затрат, алгоритм оценки частоты широкополосного сигнала.

\noindent\centerline{\bf{Основные результаты и выводы}}
В ходе диссертационного исследования получены следующие результаты:
\begin{enumerate}
\item Разработан алгоритм на основе параметрического метода оценки частоты для одного источника с широкополосным сигналом.
\item Усовершенствован алгоритм итеративного вычисления автокорреляционной функции, что позволяет использовать его в приемниках
	реального времени.
\item Разработан алгоритм оценки параметров широкополосного сигнала на основе алгоритма Delay and Multiply Approach с использованием
	предложенного усовершенствованного итеративного алгоритма вычисления автокорреляционной функции и параметрического
	метода оценки частоты. Данное решение имеет более высокую точность оценки в сравнении с традиционным
	параллельным коррелятором, в то же время оценка параметра может быть получена за меньшее количество итераций.
\item Произведено имитационное моделирование предложенного алгоритма для проверки положений, выносимых на защиту.
\item Произведено обоснование актуальности и возможности применения параметрического метода оценки частоты для сигналов
	с расширенным спектром.
\item Отражены возможные направления дальнейших исследований в области применения параметрического анализа в системах
	с расширенным спектром.
\end{enumerate}

В качестве дальнейших исследований можно выделить несколько направлений. Разработку алгоритма выбора порога, как на выходе алгоритма Delay and Multiply Approach,
так и на выходе АР модели. Так же актуальным является сравнение и анализ применение других алгоритмов оценки параметров АР модели. Анализ полной вероятности
детектирования от приемного тракта синхронизации в модуле ФАПЧ.

\newpage
