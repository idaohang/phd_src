\subsection{Алгоритм delay and multiply approach}
\label{ssec:dma}

Алгоритм был предаставлен в книге и статье американского ученого Дж.
Цуя \cite{lin_dma, tsui}

Пусть входной комплексный сигнал описывается формулой \ref{eq:dma_lo_signal}:

\begin{center}
\begin{equation}
	\label{eq:dma_lo_signal}
	s(t)=A(t)e^{j2{\pi}f_{0}t}
\end{equation}
\end{center}

где $A(t)$ - амплитуда, а $f_{0}$- частота несущей сигнала.

Если входящий комплексный сигнал имеет задержку $\tau$, то данный
сигнал будет описываться формулой: 

\begin{center}
\begin{equation}
	\label{eq:dma_signal}
	s(t-\tau)=A(t-\tau)e^{j2{\pi}f_{0}(t-\tau)}
\end{equation}
\end{center}

Получим новый сигнал путем умножения \ref{eq:dma_lo_signal} и \ref{eq:dma_signal}:

\begin{center}
\begin{eqnarray}
	s_{n}(t) & = & s(t)s(t-\tau)^{*}=\nonumber \\
	 & = & A(t)A(t-\tau)e^{j2\pi f_{0}t}e^{j2\pi f(t-\tau)}=\label{eq:dma}\\
	 & = & A(t)A(t-\tau)e^{j2\pi f_{0}\tau}\nonumber 
\end{eqnarray}

\par\end{center}

Из формулы \ref{eq:dma} видно, что полученный сигнал не зависит от
задержки $\tau$. Остается найти фазу ПСП. Референсный сигнал
$A(t)A(t-\tau)$ используется для корреляции с новым кодом, который
получен по формуле \ref{eq:dma} - умножением принятого сигнала и его задержанной
копии. Когда фаза ПСП найдена, поиск сводится к одномерному поиску
частоты. Данный метод позволяет уменьшить количество вычислений, путем
сведения задачи поиска в двух измерения: по фазе кода и частоте; к
задаче поиска только по частоте. Этот метод позволяет существенно
сэкономить вычислительные ресурсы при обнаружении сигнала заданного
спутника, но, вместе с тем, операция умножения повышает шум в процессе.

\subsubsection{Aнализ изменения ОСШ при использовании алгоритма DMA}
\label{sssec:dma_noise}

Воспользуемся математическим аппаратом теории вероятностей. Необходимый математический аппарат
рассотрен в \cite{ventcel}, для более подробного изучения можно ознакомиться с \cite{feller}.

Добавим в формулу \ref{eq:dma} АБГШ:

\begin{center}
\begin{eqnarray}
	s_{n}(t) & = & (s(t)+n_{1}(t))(s(t-\tau)+n_{2}(t))^{*}=\nonumber \\
	 & = & A(t)A(t-\tau)e^{j2{\pi}f_{0}{\tau}}+\nonumber \\
	 & + & A(t)e^{j2{\pi}f_{0}t}n_{2}(t)+\label{eq:dma_noise}\\
	 & + & A(t-\tau)e^{j2{\pi}f_{0}(t-\tau)}n_{1}(t)+\nonumber \\
	 & + & n(t)^{2}\nonumber 
\end{eqnarray}
\end{center}

где
$A(t)A(t-\tau)e^{j2{\pi}f_{0}{\tau}}$ - новая ПСП, а
$A(t)e^{j2{\pi}f_{0}t}n_{2}(t)+A(t-\tau)e^{j2{\pi}f_{0}(t-\tau)}n_{1}(t) + n(t)^{2}$ -
шумовая компонента.

Свойства дисперсии случайных величин рассмотрены в \cite{ventcel}. Дисперсия суммы 
независимых случайных величин приведена на формуле \ref{eq:var_add_full}:

\begin{center}
\begin{equation}
	\label{eq:var_add_full}
	D[\sum\limits_{i=1}^{n}{X_i}]=\sum\limits_{i=1}^{n}{D[X_i]} + 2\sum\limits_{i<j}{K_{ij}}
\end{equation}
\end{center}

Для некореллированных случайных величин можно формулу \ref{eq:var_add_full} переписать как \ref{eq:var_add}:

\begin{center}
\begin{equation}
	\label{eq:var_add}
	D[\sum\limits_{i=1}^{n}{X_i}]=\sum\limits_{i=1}^{n}{D[X_i]}
\end{equation}
\end{center}

Дисперсия произведения независимых случайных величин может быть представлена как \ref{eq:var_mult}:
\begin{center}
\begin{equation}
	\label{eq:var_mult}
	D[\prod\limits_{i=1}^{n}{X_i}]=\prod\limits_{i=1}^{n}{(D_i + m_{i}^{2})} - \prod\limits_{i=1}^{n}{m_{i}^{2}}
\end{equation}
\end{center}

Вернемся к рассмотрению формулы \ref{eq:dma_noise}.

\subsubsection{Результаты моделирования}
\label{sssec:dma_simulate}

\newpage
