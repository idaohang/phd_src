\paragraph{Оптимизация области поиска - алгоритм delay and multiply approach}
\label{ssec:dma}

Алгоритм был предаставлен в книге и статье американского ученого Дж.
Цуя \cite{lin_dma, tsui}

Преобразуем модель сигнала \ref{eq:gps_model_1}, взяв 1 сигнал из выборки, тогда индекс $k$ можно опустить:
\begin{center}
\begin{equation}
	\label{eq:dma_lo_signal}
	s(t)=C(t)e^{j2{\pi}f_{0}t}
\end{equation}
\end{center}
где $C(t)$ - амплитуда, а $f_{0}$- частота несущей сигнала.

Если входящий комплексный сигнал имеет задержку $\tau$, то данный
сигнал будет описываться формулой: 

\begin{center}
\begin{equation}
	\label{eq:dma_signal}
	s(t-\tau)=C(t-\tau)e^{j2{\pi}f_{0}(t-\tau)}
\end{equation}
\end{center}

Получим новый сигнал путем умножения \ref{eq:dma_lo_signal} и \ref{eq:dma_signal}:

\begin{center}
\begin{eqnarray}
	s_{n}(t) & = & s(t)s(t-\tau)^{*}=\nonumber \\
	 & = & C(t)C(t-\tau)e^{j2\pi f_{0}t}e^{j2\pi f(t-\tau)}=\label{eq:dma}\\
	 & = & C(t)C(t-\tau)e^{j2\pi f_{0}\tau}\nonumber 
\end{eqnarray}

\par\end{center}

Из формулы \ref{eq:dma} видно, что полученный сигнал не зависит от
задержки $\tau$. Остается найти фазу ПСП. Референсный сигнал
$C(t)C(t-\tau)$ используется для корреляции с новым кодом, который
получен по формуле \ref{eq:dma} - умножением принятого сигнала и его задержанной
копии. Когда фаза ПСП найдена, поиск сводится к одномерному поиску
частоты. Данный метод позволяет уменьшить количество вычислений, путем
сведения задачи поиска в двух измерения: по фазе кода и частоте; к
задаче поиска только по частоте. Этот метод позволяет существенно
сэкономить вычислительные ресурсы при обнаружении сигнала заданного
спутника, но, вместе с тем, операция умножения повышает шум в процессе.

\subparagraph{Aнализ изменения ОСШ при использовании алгоритма DMA}
\label{sssec:dma_noise}

Воспользуемся математическим аппаратом теории вероятностей. Необходимый математический аппарат
рассмотрен в \cite{ventcel}, для более подробного изучения можно ознакомиться с \cite{feller}.

Добавим в формулу \ref{eq:dma} АБГШ:

\begin{center}
\begin{eqnarray}
	s_{n}(t) & = & (s(t)+n_{1}(t))(s(t-\tau)+n_{2}(t))^{*}=\nonumber \\
	 & = & C(t)C(t-\tau)e^{j2{\pi}f_{0}{\tau}}+\nonumber \\
	 & + & C(t)e^{j2{\pi}f_{0}t}n_{2}(t)+\label{eq:dma_noise_1}\\
	 & + & C(t-\tau)e^{j2{\pi}f_{0}(t-\tau)}n_{1}(t)+\nonumber \\
	 & + & n(t)^{2}\nonumber 
\end{eqnarray}
\end{center}

где
$C(t)C(t-\tau)e^{j2{\pi}f_{0}{\tau}}$ - новая ПСП, умноженная на константу, а
$C(t)e^{j2{\pi}f_{0}t}n_{2}(t)+C(t-\tau)e^{j2{\pi}f_{0}(t-\tau)}n_{1}(t) + n(t)^{2}$ -
шумовая компонента.

Свойства дисперсии случайных величин рассмотрены в \cite{ventcel}. Дисперсия суммы 
независимых случайных величин приведена на формуле \ref{eq:var_add_full}:

\begin{center}
\begin{equation}
	\label{eq:var_add_full}
	D[\sum\limits_{i=1}^{n}{X_i}]=\sum\limits_{i=1}^{n}{D[X_i]} + 2\sum\limits_{i<j}{K_{ij}}
\end{equation}
\end{center}

Для некореллированных случайных величин можно формулу \ref{eq:var_add_full} переписать как \ref{eq:var_add}:

\begin{center}
\begin{equation}
	\label{eq:var_add}
	D[\sum\limits_{i=1}^{n}{X_i}]=\sum\limits_{i=1}^{n}{D[X_i]}
\end{equation}
\end{center}

Дисперсия произведения независимых случайных величин может быть представлена как \ref{eq:var_mult}:
\begin{center}
\begin{equation}
	\label{eq:var_mult}
	D[\prod\limits_{i=1}^{n}{X_i}]=\prod\limits_{i=1}^{n}{(D_i + m_{i}^{2})} - \prod\limits_{i=1}^{n}{m_{i}^{2}}
\end{equation}
\end{center}

Вернемся к рассмотрению формулы \ref{eq:dma_noise_1}. Известно, что дисперсия ПСП Гоулда равна 1, учитывая \ref{eq:var_mult}
и тот факт что математическое ожидание ПСП Гоулда равно 0. Математическое ожидаине и дисперсия гармонического сигнала в
представленном виде так же равны 0 и 1. На основе этого получаем:
\begin{center}
\begin{eqnarray}
	\label{eq:dma_noise_2}
	D[C(t)e^{j2{\pi}f_{0}t}n_{2}(t)] & = & D[n_{2}(t)] \nonumber \\
	D[C(t-\tau)e^{j2{\pi}f_{0}(t-\tau)}n_{1}(t)] & = & D[n_{1}(t)]
\end{eqnarray}
\end{center}

Учитывая \ref{eq:dma_noise_2} и теоремы о математическом ожидании и дисперсии, перепишем \ref{eq:dma_noise_1}:
\begin{center}
\begin{eqnarray}
	D[s_{n}(t)] & = & D[(s(t)+n_{1}(t))(s(t-\tau)+n_{2}(t))^{*}]=\nonumber \\
	& = & D[n(t)^{2}] + 2D[n(t)] + 1 \label{eq:dma_noise_3} \\
\end{eqnarray}
\end{center}

\subparagraph{Результаты моделирования}
\label{sssec:dma_simulate}

\subparagraph{Действительный сигнал}

Данный подход является достаточно интересным, но работает только с комплексным сигналом. Это ограничение можно
обойти, преобразуя входной сигнал из действительного в комплексный, используя преобразование Гильберта.
Так же можно специальным образом выбирать ${\tau}$.

При детектировании действительного сигнала возникают дополнительные сложности. Рассмотрим действительный сигнал:
\begin{center}
\begin{equation}
	\label{eq:dma_real1}
	s(t) = C_s(t) \sin{2\pi ft}
\end{equation}
\end{center}

Задерженная версия может быть соответственно описана как:
\begin{center}
\begin{equation}
	\label{eq:dma_real2}
	s(t - \tau) = C_s(t-\tau) \sin{[2\pi f(t-\tau)]}
\end{equation}
\end{center}

Произведение формул \ref{eq:dma_real1} и \ref{eq:dma_real2} равно \cite{tsui}:
\begin{center}
\begin{equation}
	\label{eq:dma_real3}
	s(t)s(t - \tau) = \frac{C_n(t)}{2} \{\cos (2\pi f \tau) - \cos [2 \pi f (2t - \tau)]\}
\end{equation}
\end{center}

В формуле \ref{eq:dma_real3} 2 компоненты, постоянная и высокочастотная. Высокочастотная компонента
может быть отфильтрована ФВЧ. Задержку ${\tau}$ необходимо выбирать исходя из компоненты ${\left| \cos (2\pi f \tau) \right|}$,
максимизируя значение. Нетрудно получить формулу для выбора ${\tau}$: ${\tau = \frac{1}{f} / \frac{1}{f_{samping}}}$.

\subparagraph{Недостатки}

Одним из основных недостатков является то, что данный алгоритм работает только с комплексным сигналом. Модификации
для работы с действительным сигналам ведут к дополнительным вычислительным затратам или потере точности. Для
преобразования Гильберта требуются допольнительные вычислительные затраты, а в формуле \ref{eq:dma_real3}, так
как ${\tau}$ заранее выбранная константа (${f\tau \approx 1}$), допплеровское смещение частоты будет снижать мощность
сигнала \ref{eq:dma_real3}, т.е.
\begin{center}
\begin{equation}
	\left| \cos (2\pi f \tau) \right| >= \left| \cos (2\pi (f - f_d) \tau) \right|
\end{equation}
\end{center}

\newpage
