\subsection{Алгоритм delay and multiply approach}

Алгоритм был предаставлен в книге и статье американского ученого Дж.
Цуя \cite{lin_dma, tsui}

Пусть входной комплексный сигнал описывается формулой:

\begin{center}
\begin{equation}
	s(t)=A(t)e^{j2{\pi}f_{0}t}\label{eq:dma_lo_signal}
\end{equation}
\end{center}

где $A(t)$ - амплитуда, а $f_{0}$- частота сигнала.

Если входящий комплексный сигнал имеет задержку $\tau$, то данный
сигнал будет описываться формулой: 

\begin{center}
\begin{equation}
	\label{eq:dma_signal}
	s(t-\tau)=A(t-\tau)e^{j2{\pi}f_{0}(t-\tau)}
\end{equation}
\end{center}

Получим новый сигнал путем умножения \ref{eq:dma_lo_signal} и \ref{eq:dma_signal}:

\begin{center}
\begin{eqnarray}
s_{n}(t) & = & s(t)s(t-\tau)^{*}=\nonumber \\
 & = & A(t)A(t-\tau)e^{j2\pi f_{0}t}e^{j2\pi f(t-\tau)}=\label{eq:dma}\\
 & = & A(t)A(t-\tau)e^{j2\pi f_{0}\tau}\nonumber 
\end{eqnarray}

\par\end{center}

Из формулы \ref{eq:dma} видно, что полученный сигнал не зависит от
задержки $\tau$. Остается найти фазу C/A кода. Референсный сигнал
$A(t)A(t-\tau)$ используется для корреляции с новым кодом, который
получен по формуле (3) - умножением принятого сигнала и его задержанной
копии. Когда фаза C/A кода найдена, поиск сводится к одномерному поиску
частоты. Данный метод позволяет уменьшить количество вычислений, путем
сведения задачи поиска в двух измерения: по фазе кода и частоте; к
задаче поиска только по частоте. Этот метод позволяет существенно
сэкономить вычислительные ресурсы при обнаружении сигнала заданного
спутника, но, вместе с тем, операция умножения повышает шум в процессе.

Анализ шума алгоритма DMA 

\begin{center}
\begin{eqnarray}
	s_{n}(t) & = & (s(t)+n_{1}(t))(s(t-\tau)+n_{2}(t))^{*}=\nonumber \\
	 & = & A(t)A(t-\tau)e^{j2{\pi}f_{0}{\tau}}+\nonumber \\
	 & + & A(t)e^{j2{\pi}f_{0}t}n_{2}(t)+\label{eq:dma_noise}\\
	 & + & A(t-\tau)e^{j2{\pi}f_{0}(t-\tau)}n_{1}(t)+\nonumber \\
	 & + & n(t)^{2}\nonumber 
\end{eqnarray}
\end{center}

где $A(t)A(t-\tau)e^{j2{\pi}f_{0}{\tau}}$ - новый C/A код, а $n(t)^{2}+A(t)e^{j2{\pi}f_{0}t}n_{2}(t)+A(t-\tau)e^{j2{\pi}f_{0}(t-\tau)}n_{1}(t)$-
шумовая компонента.

Поскольку корреляция производится с \textquotedbl{}новым\textquotedbl{}
кодом $A(t)A(t-\tau)$, компонента $A(t)e^{j2{\pi}f_{0}t}n_{2}(t)+A(t-\tau)e^{j2{\pi}f_{0}(t-\tau)}n_{1}(t)$
будет являеться шумом,$n(t)^{2}$ - квадрат АБГШ.

\newpage
