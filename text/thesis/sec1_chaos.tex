\subsection{Детектирование слабого сигнала с помощью осциллятора Дуффинга}
\label{ssec:duffing}

Детектирование сигналов с расширенным спекторм (в частности сигналов системы Navstar GPS) с помощью осциллятора Дуффинга
достаточно новое направление в исследованиях по данной тематике. В частности
\cite{chaos_chen, chaos_cambridge, chaos_huang, chaos_song}. Так же является интересной более ранняя статья не рассматривающая GPS
\cite{chaos_wang}.

Осциллятор Дуффинга может быть описан уравнением \ref{eq:duffing}:

\begin{center}
\begin{equation}
	\label{eq:duffing}
	mx'' + cx' + k_{1}x + k_{2}x^3 = F_{0}\cos(\omega{t})
\end{equation}
\end{center}

где
$m$- масса,
$c$ - коэффициент диссипации,
$x$ - состояние осциллятора,
$k_1$ и $k_2$ - линейный и нелинейный коэффициенты соответственно.
$F_{0}\cos(\omega{t})$ - внешнее воздействие,

\newpage
