\subsection{Методы оптимизации алгоритмов}
\subsubsection{Коррелятор}

\paragraph{Параллельный коррелятор}
\label{sec1_fft}
Вычисление циклической свертки через дискретное преобразование Фурье (ДПФ) является достаточно популярным методом
в программных приемниках, так как позволяет существенно 
уменьшить количество элементарных операций при вычислении. Но как показано в \cite{tsui, oppenheim} можно достаточно просто
перейти от свертки к циклической корреляции. Так как этот метод является самым популярным в программных приемниках, рассмотрим его
подробнее.

Обозначим импульсный отклик системы через $h(n)$, а через ${x(n)}$ - входной сигнал. Тогда выходной сигнал в дискретном
временной домене можно записать как:

\begin{equation}
	\label{eq:fft_conv}
	y(n)=\sum\limits_{m=0}^{N-1}{x(m)h(n-m)}
\end{equation}

Стоит отметить, что в \ref{eq:fft_conv} сдвиг во времени является циклическим, поскольку дискретные операции являются циклическими.
Возьмем ДПФ от \ref{eq:fft_conv}

\begin{center}
\begin{eqnarray}
	\label{eq:fft_conv_fft}
	Y(k) & = & \sum\limits_{n=0}^{N-1}\sum\limits_{m=0}^{N-1}{x(m)h(n-m)e^{(-j2\pi{kn})/N}}=\nonumber \\
	& = & \sum\limits_{m=0}^{N-1}{x(m)}[\sum\limits_{n=0}^{N-1}h(n-m)e^{(-j2\pi{(n-m)}k)/N}]e^{(-j2\pi{m}k)/N}=\\
	& = & H(k)\sum\limits_{m=0}^{N-1}e^{(-j2\pi{m}k)/N} = X(k)H(k)\nonumber 
\end{eqnarray}
\end{center}


Из уравнения \ref{eq:fft_conv_fft} легко видеть, что это не линейная свертка. В линейной свертке для $N$ входного
сигнала и свертки результатом будет $2N-1$ точек. А в уравнении выше, результатом является всего $N$ точек.
Это является результатом циклической природы ДПФ.

Алгоритм детектирования не использует свертку, он использует корреляцию, которая отличается от свертки. Корреляция
между $x(n)$ и $h(n)$ может быть записана как:

\begin{equation}
	\label{eq:fft_corr}
	y(n) = \sum\limits_{m=0}^{N-1}{x(m)h(n+m)}
\end{equation}
Единственным отличаем между \ref{eq:fft_conv} и \ref{eq:fft_corr} является знак перед $m$ в ${h(n+m)}$.
В случае детектирования сигнала $h(n)$ является локальной копией сигнала, а не импульсной характеристикой.
Произведем ДПФ над $z(n)$:

\begin{center}
\begin{eqnarray}
	\label{eq:fft_corr_fft}
	Z(k) & = & \sum\limits_{n=0}^{N-1}\sum\limits_{m=0}^{N-1}{x(m)h(n+m)e^{(-j2\pi{kn})/N}}=\nonumber \\
	& = & \sum\limits_{m=0}^{N-1}{x(m)}[\sum\limits_{n=0}^{N-1}h(n+m)e^{(-j2\pi{(n+m)}k)/N}]e^{(j2\pi{m}k)/N}=\\
	& = & H(k)\sum\limits_{m=0}^{N-1}e^{(j2\pi{m}k)/N} = X(k)H^{-1}(k)\nonumber 
\end{eqnarray}
\end{center}
где ${X^{-1}(k)}$ - обратное ДПФ. Уравнение \ref{eq:fft_corr_fft} можно записать как:

\begin{equation}
	\label{eq:fft_corr_fft_rev}
	Y(k) = \sum\limits_{n=0}^{N-1}\sum\limits_{m=0}^{N-1}{x(n+m)h(m)e^{(-j2\pi{kn})/N}}=X^{-1}(k)H(k)
\end{equation}

Если сигнал $x(n)$ реальный, то $x(n) = x^*(n)$, где * - операция комплексного сопряжения. Используя данное соотнешение,
значение $Z(k)$ может быть записано:
\begin{equation}
	\label{eq:fft_magnitude}
	|Z(k)|=|H^*(k)X(k)|=|H(k)X(k)^*|
\end{equation}
Данное соотношение может быть использовано для нахождения значения циклической корреляции между входным сигналом и 
локальной копией.

\newpage
