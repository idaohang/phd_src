\subsection{Вспомогательный способ и система для обнаружения беспроводных сигналов мобильными устройствами}

В процессе диссертационных исследований вопроса оптимизации области неопределенности при детектировании широкополосного сигнала была
подана заявка на патент \cite{patent_my}. Данная заявка на изобретение не ориентируется на конкретный алгоритм детектирования, которым
может выступать как традиционный коррелятор, так и развиваемые в данной диссертации подходы.

В формулу изобретения входит:
\begin{itemize}
	\item Способ получения вспомогательной информации о различных источниках беспроводного сигнала пользовательскими устройствами и последующего взаимообмена
	полученной информацией между указанными пользовательскими устройствами, находящимися на близком расстоянии, отличающийся тем, что включает в себя:
		\subitem отправление посредством вспомогательного модуля пользовательского устройства широковещательных сообщений с запросом для получения
		вспомогательной информации из общедоступной беспроводной сети низкой мощности и получение широковещательных сообщений с вспомогательной информацией
		из общедоступной беспроводной сети низкой мощности;
		\subitem проверка полученных широковещательных сообщений на наличие вспомогательной информации о желаемых источниках беспроводной связи;
		\subitem управление при помощи вспомогательного модуля поиском источников беспроводной связи и включением/выключением трансивера пользовательского
			устройства согласно полученной вспомогательной информации, используемой для уменьшения области неопределенности параметров беспроводного сигнала;
		\subitem сохранение вспомогательной информации в локальном кэше пользовательского устройства и ее непрерывное обновление;
		\subitem предоставление вспомогательной информации из локального кэша пользовательского устройства в широковещательном сообщении через общедоступную
			беспроводную сеть низкой мощности. 
	\item Способ по п. 1, отличающийся тем, что указанная беспроводная сеть низкой мощности выполнена на основании технологии Bluetooth, или Irda, или UWB, или Z-Wave,
		или WLAN, или ZigBee, или другой технологии, применимой в сети низкой мощности и короткого расстояния. 
	\item Способ по п. 1, отличающийся тем, что вспомогательная информация содержит список доступных источников беспроводной связи, таких как GNSS, Wi-Fi,
		CDMA, GPS, GSM и другие, и/или частоту определенного источника беспроводной связи и/или другие конфигурации источников беспроводной связи.
	\item Пользовательское устройство в беспроводной сети низкой мощности, отличающееся тем, что содержит:
		\subitem системный контур, включающий, по меньшей мере, центральное процессорное устройство и память; 
		\subitem прикладной контур, работающий с конфигурациями операционной системой и приложениями;
		\subitem коммуникационный контур, включающий беспроводные трансиверы, и управляемый вспомогательным модулем, который
		посылает широковещательные сообщения с запросом и получает широковещательные сообщения для получения вспомогательной информации из беспроводной
		сети низкой мощности, и включает и/или выключает беспроводные трансиверы согласно конфигурации пользовательского устройства и полученной вспомогательной информации.
	\item Пользовательское устройство по п. 4, отличающееся тем, что указанный вспомогательный модуль содержит программную часть, которая выключает определенный беспроводной
		трансивер, если вспомогательная информация сообщает, что желаемый беспроводной сигнал отсутствует, или продолжает процесс получения беспроводного сигнала без
		априорной информации, не выключая определенный трансивер.
	\item Пользовательское устройство по п. 4, отличающееся тем, что указанный вспомогательный модуль может сохранять вспомогательную информацию в локальном кэше
		пользовательского устройства и предоставлять вспомогательную информацию другим пользовательским устройствам в широковещательных сообщениях через беспроводную сеть
		низкой мощности.
	\item Вспомогательная система, включающая, по меньшей мере, два пользовательских устройства, находящихся на близком расстоянии друг от друга, которые могут получать и
		предоставлять вспомогательную информацию о различных источниках беспроводной связи в широковещательных сообщениях через общедоступную беспроводную сеть низкой мощности. 
	\item Вспомогательная система по п. 7, отличающаяся тем, включает пользовательские устройства по п.п. 4-6. 
\end{itemize}
