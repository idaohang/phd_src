\addcontentsline{toc}{section}{ВВЕДЕНИЕ}
\section*{ВВЕДЕНИЕ}

Большое количество современных систем являются беспроводными. Простота развертывания, мобильность, относительно низкая
стоимость - вот основные преимущества беспроводных систем. Количество мобильных устройств (телефоны, планшетные компьютеры
и т.д.) с каждым годом стремительно растет, только мобильных телефонов в 2011 году было 5.6 миллиарда и покрывало 79.86\%
\cite{wiki_mobilenum} населения земли. Технологии беспроводной связи глубоко проникли во все сферы жизни общества:
обеспечение безопасности с помощью RFID датчиков, предоставление доступа в интернет по технологиями 3G, WiFi, 
сотовая связь по различным технологиям (GSM, CDMA, DAMPS). Некоторые из этих систем строятся на основе методики
расширения спектра, которая отвечает современным требованиям по мощности сигнала, а так же по безопасности передаваемых
данных. В основе таких систем лежат шумоподобные (широкополосные) сигналы - ШПС. Вместе с тем растут требования к таким
системам. Применение ШПС ставит ряд специфических задач по обработке информации, обусловленных особенностями ШПС.
Свойства характерные для ШПС, выгодно отличают данный класс систем от класса узкополосных систем, но с другой стороны
оборачивается усложнением методов обработки ШПС.

Внедрение новых технологий требует увеличение полосы частот. Разнообразие технологий беспроводной передачи данных среди
гражданских и военных систем ведет к перегрузке каналов связи и все более высоким требованиям к скорости передачи
данных. С учётом данных требований применение систем передачи информации с ШПС становится все более востребованным.

Принимая во внимание географические размеры России и стратегическую важность обладания собственными системами спутникового
позиционирования, правительство Российский Федерации уделяет особое внимание разработке собственной системы
глобального спутникового позиционирования ГЛОНАСС. Обладание собственными технологиями СНС государство может обезопасить
себя в случае военных конфликтов от ограничения применения американской системы СНС Navstar GPS в зоне конфликта.

Разработка систем позволяющий работать с несколькими различными СНС позволит повысить точность определения координат
в сложных условиях города. Сложность детектирования сигнала и определения координат обусловлено наличием плотной
застройки многоэтажными постройками. В городских условиях задача подавления интерференционной помехи становится
актуальной. Спектр интерференционной помехи не является белым, а фильтрация и компенсация цветного шума
требует разработки специальных алгоритмов.

Разработка новых цифровых процессоров позволяет развивать подходы, которые еще 10-15 лет назад были бесперспективными.
В данной работе развиваются подходы на основе построения параметрической модели ШПС. Невозможность использования
методов требующих вычислений с высокой точностью в приемниках реального времени
10-15 лет назад была обусловлена слабой производительностью процессоров и микроконтроллеров, а так же существенной
стоимостью процессоров с модулем для операций с числами с плавающей точкой. Современное развитие цифровых технологий делает 
возможным применение параметрических методов оценки спектра взамен традиционного подхода основанного на непараметрического
анализа спектра.

Основа теории систем связи с ШПС была заложена в работах В.А. Котельникова и К. Шеннона \cite{kotelnikov-1956, shennon-63}.
России в этой области занимались В.И. Борисов, В.Б. Пестряков, В.И. Журавлев„ Л.Е.  Варакин, В.Е. Гантмахер и др.

Изначально методы расширенного спектра применялись при разработке военных систем управления и связи \cite{sklyar}.
К концу второй мировой войны расширение спектра применялось в радиолокации для борьбы с преднамеренными помехами, а
в последствии развитие данной технологии объяснялось желанием создать помехоустойчивые системы связи.
В конце 40-х-начале 50-х годов прошлого века Мортимер Рогофф, сотрудник Международной Телефонной и Телеграфной Корпорации (США) (ITT),
провёл эксперимент по передаче информации при помощи псевдошумового сигнала \cite{sklyar}, среди отечественных ученых
в середине 30-х годов прошлого века работу об основах кодового разделения каналов написал Д.В. Агеев \cite{ageev-35}.
Первые разработки таких систем относились к военным отраслям. Данный факт объясняется рядом особенностей, которыми обладают
сигналы с расширенным спектром, в числе которых — сложность перехвата заложенной в них информации,
высокая помехоустойчивость, а также трудность обнаружения факта работы передатчика. В процессе исследований расширенному спектру
нашлось и другое применение - снижение плотности энергии, высокоточная локация, использование при множественном доступе
\cite{sklyar}

Системы связи с широкополосными сигналами занимают особое место. Их особенные свойства выделяют данный класс из других систем
связи. Высокая помехозащищенность при действии сильной помехи, кодовое разделение большого количества абонентов, прием
информации с высокой достоверностью - отличительные особенности широкополосных система. Эти черты были известно, но
уровень элементной базы и низкий уровень помех не позволяли получить развития системам данного класса. Однако развитие
элементной привело к широкому распространению данного вида сигналов. В настоящее они применяются в системах спутниковой навигации,
системах сотовой связи и др \cite{varakin-book}.

Отношение сигнал/шум (ОСШ) на входе приемника может быть очень низким. Для обеспечения высокой помехозащищенности 
в таких случаях используются ШПС с большими и сверхбольшими базами.

К созданию сложных широкополосных сигналов (СШС) привело решение ряда проблем при развитии систем передачи данных.
Первая проблема встала при разработке новых радиолокационных система. Для дальнейшего развития требовалось
решить несколько противоречий: требование высокой разрешающей способности по дальности и дальностью обнаружения
целей в импульсных РЛС, требование точного измерения скорости и высокое разрешение по дальности, требование
увеличить дальность при ограничении пиковой мощности \cite{gantmaher-book}. Решение данных задач было предложено
Ф. Вудвардом \cite{vudvord-book}. Им было показано, что дополнительным параметром является форма сигнала. Длительность сигнала
может быть больше - настолько больше, насколько это необходимо для обеспечения энергетических требований, а требование
разрешения по дальности и точности измерений определяются шириной полосы сигнала. Данные требования обеспечивается
путем сжатия импульса на стороне приемника. Вудворд сформулировал принципы: произведение эффективной полосы частот
радиосигнала на его длительность должен быть существенно больше единиц ${FT>>1}$, внутренняя структура сигнала
должна быть такой, чтобы обеспечить возможность приемнику сжатие распределенного во времени сигнала в короткий импульс,
соответствующий полосе ${F}$ \cite{vudvord-book, gantmaher-book}.

В \cite{gantmaher-book} показана связь пропускной способности канала с понятием ШПС. При ${R_e<<1}$ можно записать:
\begin{center}
\begin{equation}
	\label{eq:shennon_cdma}
	FT = \frac{1}{\log(1+R_e)},
\end{equation}
\end{center}
где ${R_e}$ - ОСШ, ${F}$ - эффективная полоса частот, ${T}$ - длительность.

Стоить отметить, что при ${R_e<<1}$, левая часть выражения \ref{eq:shennon_cdma} стремится к бесконечности, а значит
ШПС позволяет обеспечить теоретически неограниченную достоверность передачи информации. Второе важное свойство
ШПС, следующее из \ref{eq:shennon_cdma} - способность работать "под шумами". Что обеспечивает скрытность
передачи информации, а с другой высокую степень уплотнения каналов связи. А значит решение современных проблем
с перегруженностью каналов связи \cite{costas-article}.

В данной работе будет рассматриваться ШПС модулированный ПСП на основе двоичной рекуррентной последовательности.
Для выделения данных из потока необходимо иметь точно синхронизированную копию ПСП, которая была использована
при модулировании сигнала на передающей стороне. Для достижения синхронизма на стороне приемника необходимо
устранить неопределенность в двух областях: неопределенность по частоте и неопределенность по фазе (задержке) ПСП.
Неопределенность по фазе ПСП обусловлена неопределенностью в расстоянии между передатчиком и приемником. Неопределенность
по частоте обусловлена в первую очередь допплеровским эффектом, а так же нестабильностью опорных генераторов в
передатчике и приемнике. После устранения неопределенности по частоте для достижения точной синхронизации
начинается процесс слежения за частотой. Неопределенность по фазе ПСП устранить не использую полный перебор
невозможно в силу корреляционных свойств ПСП. Таким образом можно заключить, что задача быстрого и эффективного
поиска ШПС является актуальной.

В данной работе рассматривается подход программного приемника (Software Defined Receiver - SDR)
\cite{akos-book, grayver-book, pany-book} для детектирования ШПС. Как уже было отражено выше, ШПС применяется во
многих системах. В данной работе будет рассматриваться сигнал СНС Navstar GPS. Данная система передачи 
информации использует ПСП Голда \cite{gold-ieee} для модулирования сигнала.

Традиционные подходы к реализации приемника СНС Navstar GPS отражены в \cite{akos-book, tsui}. 

Популярность и распространенность данной системы стимулирует исследования в области детектирования
и оценки частоты ШПС сигналов.

Существуют исследования в области применения теории хаоса - детектирование и оценка
частоты ШПС с применением осциллятора Дуффинга \cite{chaos_cambridge, chaos_chen, chaos_huang, chaos_wang}. Преимуществом
данного подхода является то, что свойства осциллятора позволяют детектировать сигналы с экстремально низким ОСШ. В то же
время в данный никто не предложил цифровое представление осциллятора Дуффинга, а это затрудняет использование данного подхода
в реальных приемниках. Таким образом данное направление является в настоящее время больше теоретическим, чем практическим.

В работах \cite{hos_petropulu, hos_zhao} предложено использовать статистики высоких порядков для подавления шума и детектирование
сигналов с низким уровнем ОСШ.

Более традиционные подходы для детектирования ШПС сигналов с низким уровнем ОСШ рассмотрены в монографии \cite{ziedan-book}.
В данной монографии рассматриваются как методы детектирования, основанные на когерентном накоплении, так и эффективные
системы слежения за частотой и фазой ПСП.

\newpage
