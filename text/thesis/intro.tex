\addcontentsline{toc}{section}{ВВЕДЕНИЕ}
\section*{ВВЕДЕНИЕ}

Большое количество современных систем являются беспроводными. Простота развертывания, мобильность, относительно низкая
стоимость - вот основные преимущества беспроводных систем. Количество мобильных устройств (телефоны, планшетные компьютеры
и т.д.) с каждым годом стремительно растет, только мобильных телефонов в 2011 году было 5.6 миллиарда и покрывало 79.86\%
\cite{wiki_mobilenum} населения земли. Технологии беспроводной связи глубоко проникли во все сферы жизни общества:
обеспечение безопасности с помощью RFID датчиков, предоставление доступа в интернет по техноолгиям 3G, WiFi, 
сотовая связь по различным технологиям (GSM, CDMA, DAMPS). Некоторые из этих систем строятся на основе методики
расширения спектра, которая отвечает современным требованиям по мощности сигнала, а так же по безопасности передаваемых
данных. В основе таких систем лежат шумоподобные (широкополосные) сигналы - ШПС. Вместе с тем растут требования к таким
системам. Применение ШПС ставит ряд специфических задач по обработке информации, обусловленных особенностями ШПС.
Свойства характерные для ШПС, выгодно отличают данный класс систем от класса узкополосных систем, но с другой стороны
оборачивается усложнением методов обработки ШПС.

Изначально методы расширенного спектра применялись при разработке военных систем управления и связи \cite{sklyar} (Глава 12.1).
К концу второй мировой войны расширение спектра применялось в радиолокации для борьбы с преднамеренными помехами, а
в последствии развитие данной технологии объяснялось желанием создать помехоустойчивые системы связи.
В конце 40-х-начале 50-х годов прошлого века Мортимер Рогофф, сотрудник Международной Телефонной и Телеграфной Корпорации (США) (ITT),
провёл эксперимент по передаче информации при помощи псевдошумового сигнала \cite{sklyar} (Глава 12.1.4.2).
Результаты его исследований продемонстрировали возможность построения системы передачи информации с шумоподобными сигналами.
Первые разработки таких систем относились к военным отраслям. Данный факт объясняется рядом особенностей, которыми обладают
сигналы с расширенным спектром, в числе которых — сложность перехвата заложенной в них информации,
высокая помехоустойчивость, а также трудность обнаружения факта работы передатчика. В процессе исследований расширенному спектру
нашлось и другое применение - снижение плотности энергии, высокоточная локация, использовние при множественном доступе
\cite{sklyar} (Глава 12.1)

\newpage
