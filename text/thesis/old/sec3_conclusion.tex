\subsection*{Выводы по главе 3}

Использование процедуры оценки параметров АР модели вместо перебора по заранее заданным значениям позволяет вести поиск сигнала
в широком диапазоне частот и определять сдвиг с более высокой точностью, устраняя необходимость дополнительного уточнения
доплеровского смещения перед запуском процедуры сопровождения сигнала.

В разделе \ref{ssec:lpc_for_one} предложен алгоритм детектирования ШПС на основе АР-модели принимаемого сигнала.
Разработанный алгоритм позволяет производить оценку частоты гармонического сигнала без использования прямого
перебора как это делается в большинстве современных алгоритмов. К недостаткам данного подхода можно отнести: 
\begin{enumerate}
	\item Сравнительно высокие вычислительные затраты. Предложенный алгоритм требует поиск гармонической
		компоненты, а так же обращения теплицевой матрицы для каждого смещения ПСП.
	\item Сильная чувствительность по отношению к интерференционным помехам: наличие
		''окрашенного'' шума приводит к значительному смещению получаемых
		оценок частоты и мощности гармонического сигнала.
\end{enumerate}

Вычислительные затраты предложенного подхода на основе АР модели второго порядка с оптимизацией перебора фазы при помощи БПФ,
составляет ${OP_{DMA\_ACF\_AR}=24NlogN+63N}$ операций умножения.

Точность оценок, полученных на основе АР модели быстро снижается при наличии сильного или окрашенного шума. Для преодоления указанных
трудностей в данной работе предлагается использовать процедуру многократной переоценки АКФ. В разделе \ref{l:ssec3_quadruple} предложена
эффективная реализация этой процедуры на основе алгоритма быстрого преобразования Фурье.

В разделе \ref{l:ssec3_dma_lpc_algo} предложен алгоритм на основе алгоритма Delay and Multiply Approach, применяемого для оценки фазы ПСП и 
АР модели для оценки частоты. Для увеличения ОСШ используется процедура многократной переоценки АКФ.
Вычислительные затраты предложенного подхода,
усовершенствованного алгоритма вычисления АКФ для компенсации шума и оценки частоты при помощи АР модели второго порядка составляет
${OP_{DMA\_ACF\_AR}=16NlogN+11N}$
операций умножения, что значительно ниже чем ${OP_{FFT\_FINE}=48NlogN+65N}$ для традиционного алгоритма параллельного коррелятора.

\newpage
