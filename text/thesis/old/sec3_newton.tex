\subsection{Использование метода Ньютона для определения ОСШ}
В разделе \ref{sssec:sec1_noise_est}, рассмотрены алгоритмы определения ОСШ, но главный недостаток этих алгоритмов в том, что
рассматривают сигнал на выходе ФАПЧ. Например, в системе Navstar GPS на выходе петли Костаса. В алгоритмах, требующих
корректировку уровня шума, необходимо осуществить оценку ОСШ (иногда мощность шума). Для алгоритма, предложенного
в разделе \ref{ssec:lpc}. Для решения данной проблемы можно использовать метод Ньютона решения систем уравнений.

Как и в разделе \ref{ssec:lpc}, при определении мощности шума с помощью алгоритма Ньютона необходимо знать фазу ПСП,
так же мы работаем в пределах одного чипа, а значит значение ${D_k(t)}$ не изменяется:

\begin{center}
\begin{equation}
	\label{eq:newton_gps_signal}
	s(t) = A \cos(\omega_{c}t) + n(t)
\end{equation}
\end{center}

В виду того, что мы хотим определить 3 компоненты сигнала: ${A}$, ${\omega}$ и ${n_(t)}$, необходимо составить 3 уравнения.
Используем АКФ сигнала приведенного в выражении \ref{eq:newton_gps_signal}:
\begin{center}
\begin{equation}
	\label{eq:newton_gps_system}
		\begin{cases}
			r_{xx}(0) = \frac{A^2}{2} + \sigma_n \\
			r_{xx}(1) = \frac{A^2}{2} \cos(\omega t) \\
			r_{xx}(2) = \frac{A^2}{2} \cos(2 \omega t)
		\end{cases}
\end{equation}
\end{center}

Приведем \ref{eq:newton_gps_system} к форме выражения \ref{eq:newton_system}:
\begin{center}
\begin{equation}
	\label{eq:newton_gps_system2}
		\begin{cases}
			\frac{A^2}{2} + \sigma_n + r_{xx}(0) = 0 \\
			\frac{A^2}{2} \cos(\omega t) + r_{xx}(1) = 0 \\
			\frac{A^2}{2} \cos(2 \omega t) + r_{xx}(2) = 0. 
		\end{cases}
\end{equation}
\end{center}

Запишем выражение \ref{eq:newton_system_3} для системы \ref{eq:newton_gps_system2}, учитывая начальное
приближение ${(\frac{r_{xx}(0)}{2}, 1, \frac{r_{xx}(0)}{2})}$:
%    J = [1 1 0;..
%        cos(z(3)*tau) 0 -tau*z(1)*sin(z(3)*tau);...
%        cos(2*z(3)*tau) 0 -2*tau*z(1)*sin(2*z(3)*tau)] ;
%z = [rxx(1)/2;rxx(1)/2;1] ;
\begin{center}
\begin{eqnarray}
	\label{eq:newton_system_3}
	J(A, \omega, \sigma_n) = 
		\left[ \begin{array}{ccc}
		1 & 1 & 0 \\
		\cos(\tau) & 0 & - \tau r_{xx}(1)/2 \sin(\tau) \\
		\cos(2\tau) & 0 & -2 \tau r_{xx}(1)/2 \sin(2 \tau)
		\end{array} \right]
\end{eqnarray}
\end{center}
\newpage
