\subsection{Неравенство Крамера-Рао в задаче оценки параметров гармонического сигнала}

В разделе \ref{s2:crlb} было рассмотрено неравенство Крамера-Рао для оценки нижней границы среднеквадратической ошибки
при использовании любых оценок параметра.
В данном разделе будет приведено данное неравенство в задаче оценки частоты
гармонического сигнала для получения границы максимально возможно точности. Приведенный вывод был взят из \cite{skon-clrb-report, rife-crlb-article}.

Пусть действительный сигнал ${s(m)}$ представлен как
\begin{center}
\begin{equation}
	\label{eq:crlb3_signal}
	s(t) = b_0 \cos(\omega_0 t + \theta_0),
	X_m = s(t_n) + W(t_m),
	Y_m = \check(s)(t_m) + \check(W)(t_m)
\end{equation}
\end{center}
${b_0}$ - амплитуда, ${\omega_0}$ - частота несущей сигнала, ${\theta_0}$ - фаза несущей сигнала, ${W(t_m)}$ - аддитивный белый гауссов шум
с дисперсией ${\sigma_R^2}$, ${t_m = t_0 + mT}$ - индекс времени, ${1/T}$ - период дескритизации,
${\check{s}(t) = b_0 \cos(\omega_0 t + \theta_0)}$, а ${\check{W}(t_m)}$ - преобразование Гильберта ${W(m)}$

Тогда если записать ${{\bf Z} = {\bf X} + j {\bf Y}}$, совместная плотность вероятности будет равна 
\begin{center}
\begin{equation}
	\label{eq:crlb3_gauss}
	f({\bf Z}; {\bf \alpha}) = \left( \frac{1}{\sigma^2 2 \pi} \right) ^M
		exp \left[ - \frac{1}{2 \sigma^2} \sum \limits_{m=0}^{M-1} (X_m - \mu_m)^2 (Y_m) - \nu_m \right]
\end{equation}
\end{center}

для вектора неизвестных параметров ${\bf \alpha}$ при условии, что все параметры сигнала неизвестны
\begin{center}
\begin{equation}
	\label{eq:crlb3_alpha}
	{\bf \alpha} = [\omega, b, \theta]^T,
	\mu_m = b \cos (\omega t_m + \theta),
	\nu_m = b \sin (\omega t_m + \theta)
\end{equation}
\end{center}

Для получения границы Крамера-Рао для несмещенной оценки параметра нужно взять соответствующий диагональный элемент
матрицы, обратной информационной матрице Фишера. Для ${f({\bf Z}; {\bf \alpha})}$ элемент матрицы будет равен
\begin{center}
\begin{eqnarray}
	\label{eq:crlb3_jacobi_ij}
	J_{ij} = \frac{1}{\sigma^2} \sum \limits_{m=0}^{M-1} 
	\left[
	\frac{\partial \mu_m}{\partial \alpha_i} \frac{\partial \mu_m}{\partial \alpha_j} \frac{\partial \mu_m}{\partial \alpha_i} \frac{\partial \mu_m}{\partial \alpha_j}
	\right]
\end{eqnarray}
\end{center}

Для общего случая, когда все параметры сигнала неизвестны матрица ${J}$ может быть записана как
\begin{center}
\begin{eqnarray}
	\label{eq:crlb3_fisher}
	J = \frac{1}{\sigma^2}
		\left[ \begin{array}{ccc}
		b_0^2 T^2 (m_0 M +2 m_0 P + Q) & 0 & b_0 T (m_0 M+P) \\
		0 & M & 0  \\
		b_0 T (m_0 M+P) & 0 & b_0^2 M  \\
		\end{array} \right]
\end{eqnarray}
\end{center}
где ${P=\sum \limits_{m=0}^{M-1}m = \frac{M(M-1)}{2}}$, а ${Q = \sum \limits_{m=0}^{M-1}m^2 = \frac{N(N-1)(2N-1)}{6}}$

Таким образом граница Крамера-Рао для оценки частоты гармонического сигнала записывается неравенством \cite{skon-clrb-report, rife-crlb-article}:
\begin{center}
\begin{eqnarray}
	\label{eq:crlb3_omega}
	D[\hat \omega] = \frac{12 \sigma^2}{b_0 T^2 M (M^2 - 1)}
\end{eqnarray}
\end{center}

