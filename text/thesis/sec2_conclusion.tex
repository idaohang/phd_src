\subsection*{Выводы по главе 2}

Особенностью ШПС является наличие ярко выраженного одиночного пика на промежуточной частоте в спектральной области сигнала после снятия расширяющего кода.
Этот факт дает основания полагать, что такие сигналы могут быть достаточно точно представлены с помощью АР модели второго порядка.

Традиционный подход оценки параметров сигнала предусматривает корреляцию входного сигнала и набора локальных копий сигнала шагом 1 кГц.
Для стационарного приемника Navstar GPS диапазон смещения частоты обусловленный допплеровским эффектом \cite{tsui} может находится в
диапазоне ${\pm 5}$ кГц. Таким образом, для получения оценки с точностью 1 кГц необходимо провести поиск в 11 ячейках.
В каждой ячейке нам необходимо ${N}$-комплексных умножений. Для наземных приемников типовое значение  ${B_L=20}$ Гц
\cite{tsui, akos-book}. Таким образом входная расстройка может быть примерно 16 Гц, в этом случан необходима стадия уточнения частоты.
Следует отметить, что заданная точность может быть достигнута даже на сигнале с достаточно низким ОСШ при оценке АР-методом после
уточнения АКФ.

\newpage
