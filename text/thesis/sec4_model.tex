\subsection{Схема эксперимента}
Для проверки развиваемых в данной работе подходов построена имитационная модель системы передачи данных с ШПС.
Модель сигнала представлена в выражении \ref{eq:gps_signal}. Так как в работе развивается 2 подхода для сигнала
с АБГШ и с интерференционной помехой. Имитационная модель позволяет выбрать необходимое количество источников сигнала.
Для проверки усовершенствованного алгоритма вычисления АКФ для компенсации окрашенной и белой аддитивной шумовой помехи
имитационная модель позволяет добавлять АБГШ к генерируемому сигналу. Таким образом имитационная модель позволяет проверить
все предлагаемые решения.

Функциональная схема системы передачи информации представлена на рисунке \ref{pic:XXX}. Как уже было отмечено, бит данныx ${D_k(t)}$
принят за констату, так при ДФМ переход нарушает гармоническую структуру входного сигнала и детектирование становится невозможным,
она учтена в модели как неизвестная начальная фаза. Несущая сигнала модулируется заданной ПСП с периодом 1023 и длительностью 1 мкс.
Частота сигнала смещена на от центральной частоты для моделирования Допплеровского смещения. Рассолгасование и нестабильность
осцилляторов на стороне передатчика учтено в допплеровском смещении. Влияние этого рассогласования крайне невилико в сравнении
со смещением частоты, обусловленным допплеровским смещением в следствии движения передающего и принимающего сегментов.

К полученному сигналу добавляется
сигнал, модулированный другой ПСП при необходимости моделирования интерференционной помехи и АБГШ. Полученная смесь подавалась
на алгоритм детектирования для определения рабочих характерстик развиваемых в работе подходов.
