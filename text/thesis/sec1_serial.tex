\subsection{Алгоритмы детектирования сигнала}

\subsubsection{Коррелятор}

\paragraph{Последовательный коррелятор}
\label{sec1_serial}
Данный алгоритм в некоторых источниках так же называется согласованным фильтром. В \cite{sklyar} рассмотрены нюансы этих двух понятий.
В данной работе мы будем использовать понятие последовательный коррелятор. Работа коррелятора описывается математической операцией
корреляции \ref{eq:serial_corr}. Сигнал коррелируется с локальной копией и на выходе коррелятора получается значение, отражающее
степень совпадения сигналов. Не трудно представить, что сигнал с хорошими корреляционными свойствами должен обладать высоким значением
корреляции когда сигналы синхронизированы и минимальным значением в любом другом случае (фаза ПСП-кода не выровнена, отсутствие сигнала).

\begin{equation}
	\label{eq:serial_corr}
	y(n)=\sum\limits_{m=0}^{N-1}{x(m)h(n+m)}
\end{equation}
где: ${x(m)}$ - принятый сигнал, а ${h(n)}$ не импульсная характеристика системы, а локальная копия сигнала.
