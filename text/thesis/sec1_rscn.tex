\section{Алгоритм оценки ОСШ "действительный сигнал - комплексный шум"}

В иностранной литературе она называется "Real Signal - Complex Noise (RSCN)". Введем перевод
"Действительный сигнал - комлексный шум (ДСКШ)".
Данный алгоритм рассмотрен в статьях \cite{badke_rscn, presti_insidegnss, presti_ieee}.

Если рассматривать идеально синхронизированный сигнал, тогда в синфазном контуре будет
находится сигнал и АБГШ, в то время как в квадратурном плече будет находится только шум,
независимый и одинаково распределенный с шумом в синфазном контуре. Данный факт может
быть использован для оценки ОСШ в программном приемнике:
\begin{center}
\begin{equation}
	\label{eq:rscn_noise_power}
	\hat{P_n} = \frac{2}{N}\sum^N_{v=1}|r_{C,Im}[v]|^2
\end{equation}
\end{center}

\begin{center}
\begin{equation}
	\label{eq:rscn_total_power}
	\hat{P}_{tot} = \frac{1}{N}\sum^N_{v=1}|r_{C}[v]|^2
\end{equation}
\end{center}

\begin{center}
\begin{equation}
	\label{eq:rscn_data_power}
	\hat{P_d} = \hat{P}_{tot} - \hat{P_n}
\end{equation}
\end{center}

\begin{center}
\begin{equation}
	\label{eq:rscn_snr}
	\hat{\lambda_C} = \frac{\hat{P_d}}{\hat{P_n}} = \frac{\hat{P}_{tot} - \hat{P_n}}{\hat{P_n}} 
\end{equation}
\end{center}

Очевидно, что данный метод является чувствительным к сдвигу фазы, который приводит к переходу энергии
в квадратурный контур. Любой остаточный сдвиг фазы ведет к возрастанию энергии шума в квадратурном
контуре (это видно из \ref{eq:rscn_noise_power}).

\newpage
