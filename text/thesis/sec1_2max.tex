\paragraph{Оптимизация решателя - алгоритм нахождения пика AКФ}

Данный алгоритм представлен в работах \cite{2max_ieee, 2max_article}. Его оригинальное название
\textquotedblleft{Peak-finding algorithm}\textquotedblright,
в данной работе введем перевод -
\textquotedblleft{Алгоритм нахождения пика}\textquotedblright (АНП). 

Алгоритм был разработан для улучшения рабочих характеристик традиционных алгоритмов рассмотренных в
\ref{sec1_serial} и \ref{sec1_fft}. Предложенный алгоритм можно разбить на несколько шагов:
\begin{itemize}
\item[Шаг 1] Подсчитать КФ, используя метод предложенный в \ref{sec1_fft}.
\item[Шаг 2] Найти главный пик КФ, найти второй пик КФ, найти среднее значение КФ.
\item[Шаг 3] Нормализовать полученные значения относительно главного пика КФ.
\item[Шаг 4] Если (максимум КФ - среднее) > ${V_{th1}}$ и (максимум КФ - 
	второй максимум КФ) > ${V_{th2}}$, тогда полученный главный пик АКФ соответсвует
	искомой фазе кода и частоте.
\end{itemize}

В статье авторов \cite{2max_ieee} предложенны следующие значения для порогов:
${V_{th1}} = 0.3$(Дб) и  ${V_{th2}} = 0.15$(Дб). Так же авторы предлагают итерационную процедуру для нахождения
фазы ПСП и частоты смещения допплера:
\begin{itemize}
\item[Шаг 1] Начать вычисление с 1мс.
\item[Шаг 2] Получить результаты АНП.
\item[Шаг 3] Если фаза ПСП и частота не могут быть найдены, увеличить время интегрироавния сигнала.
	Использовать следующие значения для интегрирования: 1мс -> 10мс -> 50мс -> 100мс -> 200мс ->
	500мс -> 1000мс
\end{itemize}


%\begin{center}
%\begin{equation}
%	\label{eq:dma_signal}
%	SNR(dB) = 10\log{\frac{max[d(n)] - mean[d(n)]}{std[d(n)]}}
%\end{equation}
%\end{center}

\newpage
