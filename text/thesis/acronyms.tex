\addcontentsline{toc}{chapter}{Список сокращений}
\chapter*{Список сокращений}
\noindent
АБГШ - аддитивный белый гауссовский шум				\\
АКП - автокорреляционная последовательность			\\
АКФ – автокорреляционная функция				\\
АНП - алгоритм нахождения пика					\\
АР - авторегрессия						\\
АРСС - авторегрессия скользящего среднего			\\
АЦП - аналогово-цифровой преобразователь			\\
БПФ - быстрое преобразование Фурье				\\
ДПФ - дискретное преобразование Фурье				\\
ДФМ - двоичная фазовая манипуляция				\\
КФ - корреляционная функция					\\
МАВ - максимальная апостериорная вероятность			\\
МШУ - малошумящий усилитель					\\
ОСШ - отношение сигнал-шум 					\\
ПЧ - промежуточная частота					\\
СД - синхронный детектор					\\
СВП - статистика высоких порядков				\\
СНРС - сигналы спутниковых радионавигационных систем		\\
СНС - спутниковая навигационная система				\\
СПМ - спектральная плотность мощности				\\
CC - скользящее среднее						\\
ПСП – псевдослучайная последовательность			\\
УГ - управляемый генератор					\\
ФАП - фазовая автоматическая подстройка				\\
ФАПЧ - фазовая автоподстройка частоты				\\
ФВЧ - фильтр высоких частот					\\
ФНЧ - фильтр низких частот					\\
ФМШПС - фазо-манипупулированный широкополосный сигнал		\\
ЦОС - цифровая обработка сигналов				\\
ШПС -  широкополосные сигналы (шумоподобные сигналы)		\\

\noindent
AGPS - Assisted GPS						\\
BPSK - Binary Phase-Shift Keying				\\
DMA - Delay and Multiply Approach				\\
FTP - file transfer protocol					\\
FPGA - field-programmable gate array 				\\
GPS - Global Positioning System					\\
HOS - Higher-order statistics					\\
NWPR - Narrowband-Wideband Power Ratio				\\
RAM - random access memory					\\
RSCN - Real Signal-Complex Noise				\\
SDK - Software Development Kit					\\
SDR - Software Defined Receiver					\\
SNV - Signal-to-Noise Variance					\\
SNR - Signal-to-Noise Rate					\\
VHDL - VHSIC Hardware Description Language			\\
WAAS - Wide Area Augmentation System				\\


\newpage
