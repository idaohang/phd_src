\subsection*{Выводы по главе 3}

Использование процедуры оценки параметров АР модели вместо перебора по заранее заданным значениям позволяет вести поиск сигнала
в широком диапазоне частот и определять сдвиг с более высокой точностью, устраняя необходимость дополнительного уточнения
доплеровского смещения перед запуском процедуры сопровождения сигнала.

Точность оценок, полученных на основе АР модели быстро снижается при наличии сильного или окрашенного шума. Для преодоления указанных
трудностей в данной работе предлагается использовать процедуру многократной переоценки АКФ. Предложена эффективная реализация
этой процедуры на основе алгоритма быстрого преобразования Фурье.

Вычислительные затраты предложенного подхода на основе АР модели второго порядка с оптимизацией перебора фазы при помощи БПФ,
составляет ${OP_{DMA\_ACF\_AR}=24NlogN+63N}$ операций умножения.

Вычислительные затраты предложенного подхода c использованием алгоритма Delay and Multiply дла определения фазы ПСП,
усовершенствованного алгоритма вычисления АКФ для компенсации шума и АР модели второго порядка составляет ${OP_{DMA\_ACF\_AR}=16NlogN+11N}$
операций умножения, что значительно ниже чем ${OP_{FFT\_FINE}=48NlogN+65N}$ для традиционного алгоритма параллельного коррелятора.

\newpage
