\subsection{Постановка задачи оценки шума}
\label{sssec:sec1_noise_est}

Задача оценки отношения сигнал-шума (ОСШ) является одной из ключевых при детектировании сигналов.
ОСШ используется в задаче определения порога детектирования рисунок \ref{pic:sec1_gnss_system}.
При превышении порога спутник считается задетектированным, если же порог
не превышен, считается, что сигнал данного спутника в данных не присутствует.

Пусть для данной задачи входной сигнал описывается соотношением \cite{presti_ieee}:
\begin{center}
\begin{equation}
	\label{eq:noise_est_signal}
	s_C[t]=\sqrt{P_d}D[n] + \sqrt{P_n}\eta[n]
\end{equation}
\end{center}
где $D[n]$ - биты навигационного сообщения, $\eta[n]=\eta_{Re} + j\eta_{Im}$ - комплексный шум,
$P_d$ - мощность сигнала, а $P_n$ - мощность шума (обе величины берутся на выходе коррелятора).
Стоит отметить, что $D[n]=a_{n}e^{j\theta_n}$, где $a_n=\pm{1}$ для сигналов с двоичной модуляцией, а
$\theta_n$ - остаточная фазовая ошибка от контура ФАПЧ слежения за частотой.
Тогда ОСШ для $s_C[t]$ можно представить как:
\begin{center}
\begin{equation}
	\label{eq:noise_est_snr}
	\lambda_C=\frac{P_d}{P_n}
\end{equation}
\end{center}
От выражения \ref{eq:noise_est_snr} можно перейти к соотношению количества шума на герц $C/N_0$:
\begin{center}
\begin{equation}
	\label{eq:noise_est_cn}
	\lambda_C=\frac{C}{N_{0}B_{eqn}}\Rightarrow\frac{C}{N_0}=\lambda_{C}B_{eqn}
\end{equation}
\end{center}
В \cite{presti_ieee} показано, что $B_{eqn}$ можно выразить:
\begin{center}
\begin{equation}
	\label{eq:noise_est_beqn}
	B_{eqn}=\frac{1}{T_{int}}
\end{equation}
\end{center}
где ${T_{int}}$ - время интегрирования.

%%%%%%%%%%%%%%%%%%%%%%%%%%%%%%%%%%%%%%%%%%%%%%%%%%%%%%
\subsubsection{Алгоритм оценки ОСШ
\textquotedblleftдействительный сигнал - комплексный шум\textquotedblright}
\label{sssec:rscn}

В иностранной литературе он называется "Real Signal - Complex Noise (RSCN)". Введем перевод
"Действительный сигнал - комлексный шум (ДСКШ)".
Данный алгоритм рассмотрен в статьях \cite{badke_rscn, presti_insidegnss, presti_ieee}.

Если рассматривать идеально синхронизированный сигнал, тогда в синфазном контуре будет
находится сигнал и АБГШ, в то время как в квадратурном плече будет находится только шум,
независимый и одинаково распределенный с шумом в синфазном контуре. Данный факт может
быть использован для оценки ОСШ в программном приемнике:
\begin{center}
\begin{equation}
	\label{eq:rscn_noise_power}
	\hat{P_n} = \frac{2}{N}\sum^N_{v=1}|r_{C,Im}[v]|^2
\end{equation}
\end{center}

\begin{center}
\begin{equation}
	\label{eq:rscn_total_power}
	\hat{P}_{tot} = \frac{1}{N}\sum^N_{v=1}|r_{C}[v]|^2
\end{equation}
\end{center}

\begin{center}
\begin{equation}
	\label{eq:rscn_data_power}
	\hat{P_d} = \hat{P}_{tot} - \hat{P_n}
\end{equation}
\end{center}

\begin{center}
\begin{equation}
	\label{eq:rscn_snr}
	\hat{\lambda_C} = \frac{\hat{P_d}}{\hat{P_n}} = \frac{\hat{P}_{tot} - \hat{P_n}}{\hat{P_n}} 
\end{equation}
\end{center}

Очевидно, что данный метод является чувствительным к сдвигу фазы, который приводит к переходу энергии
в квадратурный контур. Любой остаточный сдвиг фазы ведет к возрастанию энергии шума в квадратурном
контуре (это видно из \ref{eq:rscn_noise_power}).

%%%%%%%%%%%%%%%%%%%%%%%%%%%%%%%%%%%%%%%%%%%%%%%%%%%%%%
\subsubsection{Алгоритм Signal-to-Noise Variance}
\label{sssec:snv}

Данный алгоритм был представлен в \cite{snr_pauluzzi, snr_li}.

\begin{center}
\begin{equation}
	%\label{eq:rscn_data_power}
	\hat{P_{d}} = (\frac{1}{N} \sum \limits_{v=1}^N \left| r_{C,Re}[v] \right|)^2
\end{equation}
\end{center}

\begin{center}
\begin{equation}
	%\label{eq:rscn_data_power}
	\hat P_{tot} = \frac{1}{N} \sum \limits_{v=1}^{N} \left|r_C[v] \right| ^2
\end{equation}
\end{center}

\begin{center}
\begin{equation}
	%\label{eq:rscn_snr}
	\hat{\lambda_C} = \frac{\hat P_d}{\hat P_{tot} - \hat P_d}
\end{equation}
\end{center}

%%%%%%%%%%%%%%%%%%%%%%%%%%%%%%%%%%%%%%%%%%%%%%%%%%%%%%
\subsubsection{Алгоритм Beaulieu}
\label{sssec:beaulieu}

Данный алгоритм был представлен в статье \cite{snr_beaulieu}.

\begin{center}
\begin{equation}
	%\label{eq:rscn_data_power}
	\hat{P_{n,v}} = (\left| r_{C,Re}[v] \right| - \left| r_{C,Re}[v-1] \right|)^2
\end{equation}
\end{center}

\begin{center}
\begin{equation}
	%\label{eq:rscn_data_power}
	\hat{P_{d,v}} = \frac{1}{2}(r_{C,Re}[v]^2 + r_{C,Re}[v-1]^2)
\end{equation}
\end{center}

\begin{center}optimization
\begin{equation}
	%\label{eq:rscn_snr}
	\hat{\lambda_C} = [ \frac{1}{N} \sum \limits_{v=1}^{N} \frac{\hat P_{n,v}}{\hat P_{d,v}} ]^-1
\end{equation}
\end{center}

%%%%%%%%%%%%%%%%%%%%%%%%%%%%%%%%%%%%%%%%%%%%%%%%%%%%%%
\subsubsection{Алгоритм основанный на методе моментов}
\label{sssec:mm}

Данный алгоритм был представлен в \cite{snr_pauluzzi}.

\begin{center}
\begin{equation}
	%\label{eq:rscn_data_power}
	\hat M_2 = \frac{1}{N} \sum \limits_{v=1}^{N} \left|r_C[v] \right| ^2
\end{equation}
\end{center}

\begin{center}
\begin{equation}
	%\label{eq:rscn_data_power}
	\hat M_2 = \frac{1}{N} \sum \limits_{v=1}^{N} \left|r_C[v] \right| ^4
\end{equation}
\end{center}

\begin{center}
\begin{equation}
	%\label{eq:rscn_data_power}
	\hat P_d = \sqrt{2 \hat M^2_2 - \hat M_4} 
\end{equation}
\end{center}

\begin{center}
\begin{equation}
	%\label{eq:rscn_data_power}
	\hat P_n = \hat M_2 - \hat P_d
\end{equation}
\end{center}

\begin{center}
\begin{equation}
	%\label{eq:rscn_snr}
	\hat{\lambda_C} = \frac{\hat P_d}{\hat P_n}
\end{equation}
\end{center}

%%%%%%%%%%%%%%%%%%%%%%%%%%%%%%%%%%%%%%%%%%%%%%%%%%%%%%
\subsubsection{Алгоритм Narrowband-Wideband Power Ratio}
\label{sssec:nwpr}

Данный алгоритм был представлен в \cite{parkinson_1996}.

\begin{center}
\begin{equation}
	%\label{eq:rscn_data_power}
	WBP_k = \sum \limits_{m=1}^{M} \left|r_C[kM+m] \right| ^2, k=0,1,...(\frac{N}{M}-1)
\end{equation}
\end{center}

\begin{center}
\begin{equation}
	%\label{eq:rscn_data_power}
	NBP_k = (\sum \limits_{m=1}^{M} \left|r_{C,Re}[kM+m] \right| )^2 + (\sum \limits_{m=1}^{M} \left|r_{C,Im}[kM+m] \right| )^2
\end{equation}
\end{center}

\begin{center}
\begin{equation}
	%\label{eq:rscn_data_power}
	\hat \mu_{NP} = \frac{M}{N} \sum \limits_{k=0}^{N/M-1} \frac{NBP_k}{WBP_k}
\end{equation}
\end{center}

\begin{center}
\begin{equation}
	%\label{eq:rscn_data_power}
	\gamma = \frac{C}{N_0} = \frac{1}{T_{int}} \frac{\hat \mu_{NP} - 1}{M - \hat \mu_{NP}}
\end{equation}
\end{center}

%%%%%%%%%%%%%%%%%%%%%%%%%%%%%%%%%%%%%%%%%%%%%%%%%%%%%%
\subsubsection{Выводы}
Алгоритмы, приведенные в разделах \ref{sssec:rscn}, \ref{sssec:snv}, \ref{sssec:beaulieu}, \ref{sssec:mm}, \ref{sssec:nwpr}
подробно рассмотрены в \cite{presti_ieee}. Получены их оценки по количеству операций, а так же
исследованы свойства аппроксимации данных функций.

Следует отметить, что данные алгоритмы работают только с синхронизированным сигналом.

\newpage
