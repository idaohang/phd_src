\addcontentsline{toc}{section}{СПИСОК УСЛОВНЫХ ОБОЗНАЧЕНИЙ}
\section*{СПИСОК УСЛОВНЫХ ОБОЗНАЧЕНИЙ}
\noindent
${d(t)}$- информационный бит \\
${g(t)}$ - ПСП 	\\
${k}$ - относительный номер источника сигнала	\\
${m}$ - индекс соответствующий времени	\\
${n(m)}$ - аддитивный белый гауссов шум \\
${N}$ - количество доступных источников сигнала, модулированных ПСП одного семейства \\

\noindent
${\omega_0}$ - частота несущего колебания \\
${\tilde{\omega}_{k}}$  – относительная частота, соответствующая ${\omega_0}$ \\
${\theta_0}$ - случайная начальная фаза ${(\theta_0 \in [0, 2\pi])}$ \\
${\theta_k}$ - задержка модулирующей ПСП в точке приема \\
${\phi_k(m)}$ - случайная начальная фаза \\

%${\hat{a}_n}$ - МНК оценки коэффициентов АР-модели		\\
%${B_L}$ - шумовая полоса ФАПЧ					\\
%${d(t)}$ - расстояние между двумя близкими фазовыми траекториями	\\
%${d_i}$ - тношение расстояние между траекториями в конце ${i}$-го шага к начальному расстоянию\\
%${E[X]}$ - математическое ожидание случайной величины $X$	\\
%${D[X]}$ - дисперсия случайной величины $X$			\\
%${F}$ - матрица прямого преобра зования Фурье			\\
%${F^{-1}}$ - матрица обратного преобразования Фурье		\\
%${r_{xx}(\tau)}$ - АКФ для смещения ${\tau}$			\\
%${\bf{R_M}}$ - автокорреляционная матрица			\\
%${\omega_n}$ - собственная частота системы ФАПЧ			\\
%${\Delta \omega_m}$ - область синхронизации частоты ФАПЧ	\\
%${\zeta}$ - коэффициент демпфирования				\\
%${\lambda}$ - максимальный ляпуновский показатель		\\
%${\lambda_{i}}$ - локальный ляпуновский показатель		\\
%${R_e}$, ${R_s}$ - отношение сигнал/шум				\\
%${A}$ - амплитуда гармонического сигнала			\\
%${e(t)}$ - ошибка измерения					\\
%${\Pi}$ - порог							\\
\newpage
