\subsection{Оценки минимальной дисперсии. Неравенство Крамера-Рао}
\label{s2:crlb}

При исследовании оценок всегда возникает вопрос: нельзя ли описать границу эффективности, т.е. тот минимум (по всем возможным ${\hat \theta}$)
СКО которого равно ${E(\hat \theta -\theta)^2}$ и улучшить его невозможно? Такой минимум можно было бы использовать как базу отсчета для оценки
эффективности разрабатываемых методик оценки параметра. Такая граница может быть описана неравенством Крамера-Рао, так же в некоторой
литературе представляемое как неравенство Крамера-Рао-Фреше или еще неравенство информации. Данное неравенство широко рассмотрено в
работах по статистической радиотехнике, например \cite{aivazyan-book, levin-book, bolshakov-book}.

Пусть дана выборка ${n}$ независимых одинаково распределенных случайных величин ${\bf{X} = (X_1,...X_n)}$, плотность вероятности которых ${f(X; \theta)}$.
Можно обозначить:
\begin{center}
\begin{equation}
	\label{eq:crlb_expect}
	E \hat \theta = \int{\hat \theta} (X_1, ..., X_n)L(X_1,...,X_n;\theta)dX_1...dX_n = \theta + b_{\hat \theta}(\theta)
\end{equation}
\end{center}
где ${b_{\hat \theta}(\theta)}$ - смещение оценки ${\hat \theta}$.

Неотрицательная величина при учете независимости и одинакового распределения ${X_1,...X_n}$
\begin{center}
\begin{equation}
	\label{eq:crlb_fisher}
	{\bf I}(\theta; {\bf X}) = n \int \left( \frac{d \ln f(X;\theta)}{d \theta} \right)^2f(X;\theta)=n {\bf I} (\theta; X)
\end{equation}
\end{center}
называется информацией по Фишеру о параметре ${\theta}$, содержащийся в выборке ${\bf{X}}$

В \cite{aivazyan-book} отмечено, что если плотность ${f(X;\theta)}$ удовлетворяет условиям:
\begin{itemize}
	\item область исследуемых случайной величины ${f(X; \theta)} \ne 0$  не зависит от ${\theta}$;
	\item в \ref{eq:crlb_expect} и в ${\int L(X_1,...,X_n;\theta)dX_1...dX_n}=1$ допустимо дифференцирование по ${\theta}$ под знаком
		интеграла;
	\item величина ${\bf{I}(\theta; X)}$-количество информации Фишера не равна 0.
\end{itemize}

В таком случае для любой оценки ${\hat \theta}$ оцениваемого параметра ${\theta}$ имеет место \cite{aivazyan-book}:
\begin{center}
\begin{equation}
	\label{eq:crlb_expect_val}
	E(\hat \theta - \theta) \ge \frac{\left(1+\frac{db_{\hat \theta}(\theta)}{d \theta}\right)^2}
		{nE\left[ \left( \frac{d \ln f(X; \theta)}{d \theta}\right)\right]}
\end{equation}
\end{center}

Данное неравенство дает возможность ввести количественную меру эффективности оценок в классе регулярных (в смысле условий рассмотренных выше)
генеральных совокупностей \cite{aivazyan-book}.
